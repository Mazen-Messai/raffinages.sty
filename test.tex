\documentclass{article}
\usepackage[a4paper,margin=2cm]{geometry}
\usepackage{raffinages}
\begin{document}
\section*{Exemple d'utilisation de l'environnement \texttt{raffinage}}

\begin{raffinage}{O}{Compresser un fichier grâce à l’algorithme de Huffman}
\end{raffinage}

\begin{raffinage}{1}{Comment "Compresser un fichier grâce à l’algorithme de Huffman"}
    \commentaire{Fich et Cmpr sont des fichier, Arbr est un arbre, Tbl est un tableau
    d’arbre, et Parc et List sont des chaînes de caractère}  
    \step{Créer la table des occurnces}[Fich : out, Tbl : out]
    \step{Créer l’arbre de Huffman}[Tbl : in, Arbr : out]
    \step{Parcourir l’arbre de Huffman}[Arbr : in, Parc : out]
    \step{Créer la liste de Huffman}[Parc : in, List : out]
    \step{Compresser le fichier}[Fich : in, List : in, Cmpr : out]
\end{raffinage}

\begin{raffinage}{2}{Comment "Créer la table des occurences"}
    \step{Initialiser(Tbl)}[Tbl : out]
    \step{Ouvrir le fichier}[Fich : in]
    \begin{whilestructure}{non vide(Fich)}
        \step{Lire le caractère}[Car : out]
        \step{Indx ← Chercher(Tbl, Car)}[Car, Tbl : in, Indx : Out]
        \step{Mettree à jour la table}[Tbl : in out, Indx : in]
    \end{whilestructure}
\end{raffinage}

\begin{raffinage}{3}{Blabla}
    \begin{ifstructure}{condition}
        \step{Faire quelque chose}[Fich : in]
        \step{Faire autre chose}[Fich : in]
        \step{Faire autre chose}[Fich : in]
    \end{ifstructure}
\end{raffinage}

\end{document}